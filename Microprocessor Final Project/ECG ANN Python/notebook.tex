
% Default to the notebook output style

    


% Inherit from the specified cell style.




    
\documentclass[11pt]{article}

    
    
    \usepackage[T1]{fontenc}
    % Nicer default font (+ math font) than Computer Modern for most use cases
    \usepackage{mathpazo}

    % Basic figure setup, for now with no caption control since it's done
    % automatically by Pandoc (which extracts ![](path) syntax from Markdown).
    \usepackage{graphicx}
    % We will generate all images so they have a width \maxwidth. This means
    % that they will get their normal width if they fit onto the page, but
    % are scaled down if they would overflow the margins.
    \makeatletter
    \def\maxwidth{\ifdim\Gin@nat@width>\linewidth\linewidth
    \else\Gin@nat@width\fi}
    \makeatother
    \let\Oldincludegraphics\includegraphics
    % Set max figure width to be 80% of text width, for now hardcoded.
    \renewcommand{\includegraphics}[1]{\Oldincludegraphics[width=.8\maxwidth]{#1}}
    % Ensure that by default, figures have no caption (until we provide a
    % proper Figure object with a Caption API and a way to capture that
    % in the conversion process - todo).
    \usepackage{caption}
    \DeclareCaptionLabelFormat{nolabel}{}
    \captionsetup{labelformat=nolabel}

    \usepackage{adjustbox} % Used to constrain images to a maximum size 
    \usepackage{xcolor} % Allow colors to be defined
    \usepackage{enumerate} % Needed for markdown enumerations to work
    \usepackage{geometry} % Used to adjust the document margins
    \usepackage{amsmath} % Equations
    \usepackage{amssymb} % Equations
    \usepackage{textcomp} % defines textquotesingle
    % Hack from http://tex.stackexchange.com/a/47451/13684:
    \AtBeginDocument{%
        \def\PYZsq{\textquotesingle}% Upright quotes in Pygmentized code
    }
    \usepackage{upquote} % Upright quotes for verbatim code
    \usepackage{eurosym} % defines \euro
    \usepackage[mathletters]{ucs} % Extended unicode (utf-8) support
    \usepackage[utf8x]{inputenc} % Allow utf-8 characters in the tex document
    \usepackage{fancyvrb} % verbatim replacement that allows latex
    \usepackage{grffile} % extends the file name processing of package graphics 
                         % to support a larger range 
    % The hyperref package gives us a pdf with properly built
    % internal navigation ('pdf bookmarks' for the table of contents,
    % internal cross-reference links, web links for URLs, etc.)
    \usepackage{hyperref}
    \usepackage{longtable} % longtable support required by pandoc >1.10
    \usepackage{booktabs}  % table support for pandoc > 1.12.2
    \usepackage[inline]{enumitem} % IRkernel/repr support (it uses the enumerate* environment)
    \usepackage[normalem]{ulem} % ulem is needed to support strikethroughs (\sout)
                                % normalem makes italics be italics, not underlines
    

    
    
    % Colors for the hyperref package
    \definecolor{urlcolor}{rgb}{0,.145,.698}
    \definecolor{linkcolor}{rgb}{.71,0.21,0.01}
    \definecolor{citecolor}{rgb}{.12,.54,.11}

    % ANSI colors
    \definecolor{ansi-black}{HTML}{3E424D}
    \definecolor{ansi-black-intense}{HTML}{282C36}
    \definecolor{ansi-red}{HTML}{E75C58}
    \definecolor{ansi-red-intense}{HTML}{B22B31}
    \definecolor{ansi-green}{HTML}{00A250}
    \definecolor{ansi-green-intense}{HTML}{007427}
    \definecolor{ansi-yellow}{HTML}{DDB62B}
    \definecolor{ansi-yellow-intense}{HTML}{B27D12}
    \definecolor{ansi-blue}{HTML}{208FFB}
    \definecolor{ansi-blue-intense}{HTML}{0065CA}
    \definecolor{ansi-magenta}{HTML}{D160C4}
    \definecolor{ansi-magenta-intense}{HTML}{A03196}
    \definecolor{ansi-cyan}{HTML}{60C6C8}
    \definecolor{ansi-cyan-intense}{HTML}{258F8F}
    \definecolor{ansi-white}{HTML}{C5C1B4}
    \definecolor{ansi-white-intense}{HTML}{A1A6B2}

    % commands and environments needed by pandoc snippets
    % extracted from the output of `pandoc -s`
    \providecommand{\tightlist}{%
      \setlength{\itemsep}{0pt}\setlength{\parskip}{0pt}}
    \DefineVerbatimEnvironment{Highlighting}{Verbatim}{commandchars=\\\{\}}
    % Add ',fontsize=\small' for more characters per line
    \newenvironment{Shaded}{}{}
    \newcommand{\KeywordTok}[1]{\textcolor[rgb]{0.00,0.44,0.13}{\textbf{{#1}}}}
    \newcommand{\DataTypeTok}[1]{\textcolor[rgb]{0.56,0.13,0.00}{{#1}}}
    \newcommand{\DecValTok}[1]{\textcolor[rgb]{0.25,0.63,0.44}{{#1}}}
    \newcommand{\BaseNTok}[1]{\textcolor[rgb]{0.25,0.63,0.44}{{#1}}}
    \newcommand{\FloatTok}[1]{\textcolor[rgb]{0.25,0.63,0.44}{{#1}}}
    \newcommand{\CharTok}[1]{\textcolor[rgb]{0.25,0.44,0.63}{{#1}}}
    \newcommand{\StringTok}[1]{\textcolor[rgb]{0.25,0.44,0.63}{{#1}}}
    \newcommand{\CommentTok}[1]{\textcolor[rgb]{0.38,0.63,0.69}{\textit{{#1}}}}
    \newcommand{\OtherTok}[1]{\textcolor[rgb]{0.00,0.44,0.13}{{#1}}}
    \newcommand{\AlertTok}[1]{\textcolor[rgb]{1.00,0.00,0.00}{\textbf{{#1}}}}
    \newcommand{\FunctionTok}[1]{\textcolor[rgb]{0.02,0.16,0.49}{{#1}}}
    \newcommand{\RegionMarkerTok}[1]{{#1}}
    \newcommand{\ErrorTok}[1]{\textcolor[rgb]{1.00,0.00,0.00}{\textbf{{#1}}}}
    \newcommand{\NormalTok}[1]{{#1}}
    
    % Additional commands for more recent versions of Pandoc
    \newcommand{\ConstantTok}[1]{\textcolor[rgb]{0.53,0.00,0.00}{{#1}}}
    \newcommand{\SpecialCharTok}[1]{\textcolor[rgb]{0.25,0.44,0.63}{{#1}}}
    \newcommand{\VerbatimStringTok}[1]{\textcolor[rgb]{0.25,0.44,0.63}{{#1}}}
    \newcommand{\SpecialStringTok}[1]{\textcolor[rgb]{0.73,0.40,0.53}{{#1}}}
    \newcommand{\ImportTok}[1]{{#1}}
    \newcommand{\DocumentationTok}[1]{\textcolor[rgb]{0.73,0.13,0.13}{\textit{{#1}}}}
    \newcommand{\AnnotationTok}[1]{\textcolor[rgb]{0.38,0.63,0.69}{\textbf{\textit{{#1}}}}}
    \newcommand{\CommentVarTok}[1]{\textcolor[rgb]{0.38,0.63,0.69}{\textbf{\textit{{#1}}}}}
    \newcommand{\VariableTok}[1]{\textcolor[rgb]{0.10,0.09,0.49}{{#1}}}
    \newcommand{\ControlFlowTok}[1]{\textcolor[rgb]{0.00,0.44,0.13}{\textbf{{#1}}}}
    \newcommand{\OperatorTok}[1]{\textcolor[rgb]{0.40,0.40,0.40}{{#1}}}
    \newcommand{\BuiltInTok}[1]{{#1}}
    \newcommand{\ExtensionTok}[1]{{#1}}
    \newcommand{\PreprocessorTok}[1]{\textcolor[rgb]{0.74,0.48,0.00}{{#1}}}
    \newcommand{\AttributeTok}[1]{\textcolor[rgb]{0.49,0.56,0.16}{{#1}}}
    \newcommand{\InformationTok}[1]{\textcolor[rgb]{0.38,0.63,0.69}{\textbf{\textit{{#1}}}}}
    \newcommand{\WarningTok}[1]{\textcolor[rgb]{0.38,0.63,0.69}{\textbf{\textit{{#1}}}}}
    
    
    % Define a nice break command that doesn't care if a line doesn't already
    % exist.
    \def\br{\hspace*{\fill} \\* }
    % Math Jax compatability definitions
    \def\gt{>}
    \def\lt{<}
    % Document parameters
    \title{Arrythmia ECG Classification using ANN}
    
    
    

    % Pygments definitions
    
\makeatletter
\def\PY@reset{\let\PY@it=\relax \let\PY@bf=\relax%
    \let\PY@ul=\relax \let\PY@tc=\relax%
    \let\PY@bc=\relax \let\PY@ff=\relax}
\def\PY@tok#1{\csname PY@tok@#1\endcsname}
\def\PY@toks#1+{\ifx\relax#1\empty\else%
    \PY@tok{#1}\expandafter\PY@toks\fi}
\def\PY@do#1{\PY@bc{\PY@tc{\PY@ul{%
    \PY@it{\PY@bf{\PY@ff{#1}}}}}}}
\def\PY#1#2{\PY@reset\PY@toks#1+\relax+\PY@do{#2}}

\expandafter\def\csname PY@tok@w\endcsname{\def\PY@tc##1{\textcolor[rgb]{0.73,0.73,0.73}{##1}}}
\expandafter\def\csname PY@tok@c\endcsname{\let\PY@it=\textit\def\PY@tc##1{\textcolor[rgb]{0.25,0.50,0.50}{##1}}}
\expandafter\def\csname PY@tok@cp\endcsname{\def\PY@tc##1{\textcolor[rgb]{0.74,0.48,0.00}{##1}}}
\expandafter\def\csname PY@tok@k\endcsname{\let\PY@bf=\textbf\def\PY@tc##1{\textcolor[rgb]{0.00,0.50,0.00}{##1}}}
\expandafter\def\csname PY@tok@kp\endcsname{\def\PY@tc##1{\textcolor[rgb]{0.00,0.50,0.00}{##1}}}
\expandafter\def\csname PY@tok@kt\endcsname{\def\PY@tc##1{\textcolor[rgb]{0.69,0.00,0.25}{##1}}}
\expandafter\def\csname PY@tok@o\endcsname{\def\PY@tc##1{\textcolor[rgb]{0.40,0.40,0.40}{##1}}}
\expandafter\def\csname PY@tok@ow\endcsname{\let\PY@bf=\textbf\def\PY@tc##1{\textcolor[rgb]{0.67,0.13,1.00}{##1}}}
\expandafter\def\csname PY@tok@nb\endcsname{\def\PY@tc##1{\textcolor[rgb]{0.00,0.50,0.00}{##1}}}
\expandafter\def\csname PY@tok@nf\endcsname{\def\PY@tc##1{\textcolor[rgb]{0.00,0.00,1.00}{##1}}}
\expandafter\def\csname PY@tok@nc\endcsname{\let\PY@bf=\textbf\def\PY@tc##1{\textcolor[rgb]{0.00,0.00,1.00}{##1}}}
\expandafter\def\csname PY@tok@nn\endcsname{\let\PY@bf=\textbf\def\PY@tc##1{\textcolor[rgb]{0.00,0.00,1.00}{##1}}}
\expandafter\def\csname PY@tok@ne\endcsname{\let\PY@bf=\textbf\def\PY@tc##1{\textcolor[rgb]{0.82,0.25,0.23}{##1}}}
\expandafter\def\csname PY@tok@nv\endcsname{\def\PY@tc##1{\textcolor[rgb]{0.10,0.09,0.49}{##1}}}
\expandafter\def\csname PY@tok@no\endcsname{\def\PY@tc##1{\textcolor[rgb]{0.53,0.00,0.00}{##1}}}
\expandafter\def\csname PY@tok@nl\endcsname{\def\PY@tc##1{\textcolor[rgb]{0.63,0.63,0.00}{##1}}}
\expandafter\def\csname PY@tok@ni\endcsname{\let\PY@bf=\textbf\def\PY@tc##1{\textcolor[rgb]{0.60,0.60,0.60}{##1}}}
\expandafter\def\csname PY@tok@na\endcsname{\def\PY@tc##1{\textcolor[rgb]{0.49,0.56,0.16}{##1}}}
\expandafter\def\csname PY@tok@nt\endcsname{\let\PY@bf=\textbf\def\PY@tc##1{\textcolor[rgb]{0.00,0.50,0.00}{##1}}}
\expandafter\def\csname PY@tok@nd\endcsname{\def\PY@tc##1{\textcolor[rgb]{0.67,0.13,1.00}{##1}}}
\expandafter\def\csname PY@tok@s\endcsname{\def\PY@tc##1{\textcolor[rgb]{0.73,0.13,0.13}{##1}}}
\expandafter\def\csname PY@tok@sd\endcsname{\let\PY@it=\textit\def\PY@tc##1{\textcolor[rgb]{0.73,0.13,0.13}{##1}}}
\expandafter\def\csname PY@tok@si\endcsname{\let\PY@bf=\textbf\def\PY@tc##1{\textcolor[rgb]{0.73,0.40,0.53}{##1}}}
\expandafter\def\csname PY@tok@se\endcsname{\let\PY@bf=\textbf\def\PY@tc##1{\textcolor[rgb]{0.73,0.40,0.13}{##1}}}
\expandafter\def\csname PY@tok@sr\endcsname{\def\PY@tc##1{\textcolor[rgb]{0.73,0.40,0.53}{##1}}}
\expandafter\def\csname PY@tok@ss\endcsname{\def\PY@tc##1{\textcolor[rgb]{0.10,0.09,0.49}{##1}}}
\expandafter\def\csname PY@tok@sx\endcsname{\def\PY@tc##1{\textcolor[rgb]{0.00,0.50,0.00}{##1}}}
\expandafter\def\csname PY@tok@m\endcsname{\def\PY@tc##1{\textcolor[rgb]{0.40,0.40,0.40}{##1}}}
\expandafter\def\csname PY@tok@gh\endcsname{\let\PY@bf=\textbf\def\PY@tc##1{\textcolor[rgb]{0.00,0.00,0.50}{##1}}}
\expandafter\def\csname PY@tok@gu\endcsname{\let\PY@bf=\textbf\def\PY@tc##1{\textcolor[rgb]{0.50,0.00,0.50}{##1}}}
\expandafter\def\csname PY@tok@gd\endcsname{\def\PY@tc##1{\textcolor[rgb]{0.63,0.00,0.00}{##1}}}
\expandafter\def\csname PY@tok@gi\endcsname{\def\PY@tc##1{\textcolor[rgb]{0.00,0.63,0.00}{##1}}}
\expandafter\def\csname PY@tok@gr\endcsname{\def\PY@tc##1{\textcolor[rgb]{1.00,0.00,0.00}{##1}}}
\expandafter\def\csname PY@tok@ge\endcsname{\let\PY@it=\textit}
\expandafter\def\csname PY@tok@gs\endcsname{\let\PY@bf=\textbf}
\expandafter\def\csname PY@tok@gp\endcsname{\let\PY@bf=\textbf\def\PY@tc##1{\textcolor[rgb]{0.00,0.00,0.50}{##1}}}
\expandafter\def\csname PY@tok@go\endcsname{\def\PY@tc##1{\textcolor[rgb]{0.53,0.53,0.53}{##1}}}
\expandafter\def\csname PY@tok@gt\endcsname{\def\PY@tc##1{\textcolor[rgb]{0.00,0.27,0.87}{##1}}}
\expandafter\def\csname PY@tok@err\endcsname{\def\PY@bc##1{\setlength{\fboxsep}{0pt}\fcolorbox[rgb]{1.00,0.00,0.00}{1,1,1}{\strut ##1}}}
\expandafter\def\csname PY@tok@kc\endcsname{\let\PY@bf=\textbf\def\PY@tc##1{\textcolor[rgb]{0.00,0.50,0.00}{##1}}}
\expandafter\def\csname PY@tok@kd\endcsname{\let\PY@bf=\textbf\def\PY@tc##1{\textcolor[rgb]{0.00,0.50,0.00}{##1}}}
\expandafter\def\csname PY@tok@kn\endcsname{\let\PY@bf=\textbf\def\PY@tc##1{\textcolor[rgb]{0.00,0.50,0.00}{##1}}}
\expandafter\def\csname PY@tok@kr\endcsname{\let\PY@bf=\textbf\def\PY@tc##1{\textcolor[rgb]{0.00,0.50,0.00}{##1}}}
\expandafter\def\csname PY@tok@bp\endcsname{\def\PY@tc##1{\textcolor[rgb]{0.00,0.50,0.00}{##1}}}
\expandafter\def\csname PY@tok@fm\endcsname{\def\PY@tc##1{\textcolor[rgb]{0.00,0.00,1.00}{##1}}}
\expandafter\def\csname PY@tok@vc\endcsname{\def\PY@tc##1{\textcolor[rgb]{0.10,0.09,0.49}{##1}}}
\expandafter\def\csname PY@tok@vg\endcsname{\def\PY@tc##1{\textcolor[rgb]{0.10,0.09,0.49}{##1}}}
\expandafter\def\csname PY@tok@vi\endcsname{\def\PY@tc##1{\textcolor[rgb]{0.10,0.09,0.49}{##1}}}
\expandafter\def\csname PY@tok@vm\endcsname{\def\PY@tc##1{\textcolor[rgb]{0.10,0.09,0.49}{##1}}}
\expandafter\def\csname PY@tok@sa\endcsname{\def\PY@tc##1{\textcolor[rgb]{0.73,0.13,0.13}{##1}}}
\expandafter\def\csname PY@tok@sb\endcsname{\def\PY@tc##1{\textcolor[rgb]{0.73,0.13,0.13}{##1}}}
\expandafter\def\csname PY@tok@sc\endcsname{\def\PY@tc##1{\textcolor[rgb]{0.73,0.13,0.13}{##1}}}
\expandafter\def\csname PY@tok@dl\endcsname{\def\PY@tc##1{\textcolor[rgb]{0.73,0.13,0.13}{##1}}}
\expandafter\def\csname PY@tok@s2\endcsname{\def\PY@tc##1{\textcolor[rgb]{0.73,0.13,0.13}{##1}}}
\expandafter\def\csname PY@tok@sh\endcsname{\def\PY@tc##1{\textcolor[rgb]{0.73,0.13,0.13}{##1}}}
\expandafter\def\csname PY@tok@s1\endcsname{\def\PY@tc##1{\textcolor[rgb]{0.73,0.13,0.13}{##1}}}
\expandafter\def\csname PY@tok@mb\endcsname{\def\PY@tc##1{\textcolor[rgb]{0.40,0.40,0.40}{##1}}}
\expandafter\def\csname PY@tok@mf\endcsname{\def\PY@tc##1{\textcolor[rgb]{0.40,0.40,0.40}{##1}}}
\expandafter\def\csname PY@tok@mh\endcsname{\def\PY@tc##1{\textcolor[rgb]{0.40,0.40,0.40}{##1}}}
\expandafter\def\csname PY@tok@mi\endcsname{\def\PY@tc##1{\textcolor[rgb]{0.40,0.40,0.40}{##1}}}
\expandafter\def\csname PY@tok@il\endcsname{\def\PY@tc##1{\textcolor[rgb]{0.40,0.40,0.40}{##1}}}
\expandafter\def\csname PY@tok@mo\endcsname{\def\PY@tc##1{\textcolor[rgb]{0.40,0.40,0.40}{##1}}}
\expandafter\def\csname PY@tok@ch\endcsname{\let\PY@it=\textit\def\PY@tc##1{\textcolor[rgb]{0.25,0.50,0.50}{##1}}}
\expandafter\def\csname PY@tok@cm\endcsname{\let\PY@it=\textit\def\PY@tc##1{\textcolor[rgb]{0.25,0.50,0.50}{##1}}}
\expandafter\def\csname PY@tok@cpf\endcsname{\let\PY@it=\textit\def\PY@tc##1{\textcolor[rgb]{0.25,0.50,0.50}{##1}}}
\expandafter\def\csname PY@tok@c1\endcsname{\let\PY@it=\textit\def\PY@tc##1{\textcolor[rgb]{0.25,0.50,0.50}{##1}}}
\expandafter\def\csname PY@tok@cs\endcsname{\let\PY@it=\textit\def\PY@tc##1{\textcolor[rgb]{0.25,0.50,0.50}{##1}}}

\def\PYZbs{\char`\\}
\def\PYZus{\char`\_}
\def\PYZob{\char`\{}
\def\PYZcb{\char`\}}
\def\PYZca{\char`\^}
\def\PYZam{\char`\&}
\def\PYZlt{\char`\<}
\def\PYZgt{\char`\>}
\def\PYZsh{\char`\#}
\def\PYZpc{\char`\%}
\def\PYZdl{\char`\$}
\def\PYZhy{\char`\-}
\def\PYZsq{\char`\'}
\def\PYZdq{\char`\"}
\def\PYZti{\char`\~}
% for compatibility with earlier versions
\def\PYZat{@}
\def\PYZlb{[}
\def\PYZrb{]}
\makeatother


    % Exact colors from NB
    \definecolor{incolor}{rgb}{0.0, 0.0, 0.5}
    \definecolor{outcolor}{rgb}{0.545, 0.0, 0.0}



    
    % Prevent overflowing lines due to hard-to-break entities
    \sloppy 
    % Setup hyperref package
    \hypersetup{
      breaklinks=true,  % so long urls are correctly broken across lines
      colorlinks=true,
      urlcolor=urlcolor,
      linkcolor=linkcolor,
      citecolor=citecolor,
      }
    % Slightly bigger margins than the latex defaults
    
    \geometry{verbose,tmargin=1in,bmargin=1in,lmargin=1in,rmargin=1in}
    
    

    \begin{document}
    
    
    \maketitle
    
    

    
    \hypertarget{low-level-neural-network-for-ecg-classification}{%
\section{Low Level Neural Network for ECG
Classification}\label{low-level-neural-network-for-ecg-classification}}

In this notebook we implement a 3-layer ANN for classifying ECG signals
using PSD and Energy as input. Our three layers consist of: Layer 1: 2
neurons to input the PSD and Energy. Layer 2: In Layer 2 we used a
variable amount of neurons ,namely from one to 30 neurons and we
obtained the best metrics by using a 16 neuron layer. Layer 3: 2 neurons
to output the probability of a patient having a Normal sinus or an
arrythmia.

The Network is implemented from scratch to mimic the flow of the process
on low-level hardware.

    \begin{Verbatim}[commandchars=\\\{\}]
{\color{incolor}In [{\color{incolor}8}]:} \PY{c+c1}{\PYZsh{} Library imports}
        \PY{k+kn}{import} \PY{n+nn}{numpy} \PY{k}{as} \PY{n+nn}{np}
        \PY{k+kn}{import} \PY{n+nn}{matplotlib}
        \PY{k+kn}{import} \PY{n+nn}{csv}
        \PY{k+kn}{from} \PY{n+nn}{matplotlib} \PY{k}{import} \PY{n}{pyplot} \PY{k}{as} \PY{n}{plt}
        \PY{k+kn}{from} \PY{n+nn}{matplotlib}\PY{n+nn}{.}\PY{n+nn}{pyplot} \PY{k}{import} \PY{n}{figure}
\end{Verbatim}


    \hypertarget{loading-datasets}{%
\subsection{Loading datasets}\label{loading-datasets}}

We load the arrythmia and normal sinus datasets from
https://physionet.org/physiobank/database/\#ecg MIT-BIH and split the
data into two sets, a training set and a post-training test set. We also
perform normalization of data to decrease the training speed when using
the gradient descent optimization.

    \begin{Verbatim}[commandchars=\\\{\}]
{\color{incolor}In [{\color{incolor}9}]:} \PY{c+c1}{\PYZsh{} Load dataset and plot a scatter\PYZhy{}graph}
        \PY{n}{np}\PY{o}{.}\PY{n}{random}\PY{o}{.}\PY{n}{seed}\PY{p}{(}\PY{l+m+mi}{0}\PY{p}{)}
        \PY{n}{X1\PYZus{}init}\PY{o}{=} \PY{p}{[}\PY{p}{]}
        \PY{n}{y\PYZus{}init}\PY{o}{=} \PY{p}{[}\PY{p}{]}
        \PY{n}{max1} \PY{o}{=} \PY{l+m+mi}{0}
        \PY{n}{max2} \PY{o}{=} \PY{l+m+mi}{0}
        \PY{n}{min1} \PY{o}{=} \PY{l+m+mi}{99999999}
        \PY{n}{min2} \PY{o}{=} \PY{l+m+mi}{99999999}
        
        \PY{k}{with} \PY{n+nb}{open}\PY{p}{(}\PY{l+s+s1}{\PYZsq{}}\PY{l+s+s1}{database.csv}\PY{l+s+s1}{\PYZsq{}}\PY{p}{,} \PY{l+s+s1}{\PYZsq{}}\PY{l+s+s1}{rt}\PY{l+s+s1}{\PYZsq{}}\PY{p}{)} \PY{k}{as} \PY{n}{csvfile}\PY{p}{:}
            \PY{n}{lines} \PY{o}{=} \PY{n}{csv}\PY{o}{.}\PY{n}{reader}\PY{p}{(}\PY{n}{csvfile}\PY{p}{)}
            \PY{k}{for} \PY{n}{row} \PY{o+ow}{in} \PY{n}{lines}\PY{p}{:}
                \PY{n}{dummy} \PY{o}{=} \PY{p}{[}\PY{p}{]}
                \PY{n}{dummy}\PY{o}{.}\PY{n}{append}\PY{p}{(}\PY{n+nb}{float}\PY{p}{(}\PY{n}{row}\PY{p}{[}\PY{l+m+mi}{1}\PY{p}{]}\PY{p}{)}\PY{o}{*}\PY{l+m+mf}{1e10}\PY{p}{)}
                \PY{n}{dummy}\PY{o}{.}\PY{n}{append}\PY{p}{(}\PY{n+nb}{float}\PY{p}{(}\PY{n}{row}\PY{p}{[}\PY{l+m+mi}{2}\PY{p}{]}\PY{p}{)}\PY{o}{/}\PY{l+m+mf}{1e5}\PY{p}{)}
                \PY{n}{max1} \PY{o}{=} \PY{n+nb}{max}\PY{p}{(}\PY{n}{max1}\PY{p}{,} \PY{n}{dummy}\PY{p}{[}\PY{l+m+mi}{0}\PY{p}{]}\PY{p}{)}
                \PY{n}{max2} \PY{o}{=} \PY{n+nb}{max}\PY{p}{(}\PY{n}{max2}\PY{p}{,} \PY{n}{dummy}\PY{p}{[}\PY{l+m+mi}{1}\PY{p}{]}\PY{p}{)}
                \PY{n}{min1} \PY{o}{=} \PY{n+nb}{min}\PY{p}{(}\PY{n}{min1}\PY{p}{,} \PY{n}{dummy}\PY{p}{[}\PY{l+m+mi}{0}\PY{p}{]}\PY{p}{)}
                \PY{n}{min2} \PY{o}{=} \PY{n+nb}{min}\PY{p}{(}\PY{n}{min2}\PY{p}{,} \PY{n}{dummy}\PY{p}{[}\PY{l+m+mi}{1}\PY{p}{]}\PY{p}{)}
                \PY{n}{X1\PYZus{}init}\PY{o}{.}\PY{n}{append}\PY{p}{(}\PY{n}{dummy}\PY{p}{)}
                \PY{n}{y\PYZus{}init}\PY{o}{.}\PY{n}{append}\PY{p}{(}\PY{n+nb}{int}\PY{p}{(}\PY{n}{row}\PY{p}{[}\PY{l+m+mi}{3}\PY{p}{]}\PY{p}{)}\PY{p}{)}
        
        \PY{n}{X\PYZus{}init} \PY{o}{=} \PY{p}{[}\PY{p}{]}
        \PY{k}{for} \PY{n}{val} \PY{o+ow}{in} \PY{n}{X1\PYZus{}init}\PY{p}{:}
            \PY{n}{dumbo} \PY{o}{=} \PY{p}{[}\PY{p}{]}
            \PY{n}{v1} \PY{o}{=} \PY{p}{(}\PY{n}{val}\PY{p}{[}\PY{l+m+mi}{0}\PY{p}{]}\PY{o}{\PYZhy{}} \PY{n}{min1}\PY{p}{)}\PY{o}{/}\PY{p}{(}\PY{n}{max1}\PY{o}{\PYZhy{}}\PY{n}{min1}\PY{p}{)}
            \PY{n}{v2} \PY{o}{=} \PY{p}{(}\PY{n}{val}\PY{p}{[}\PY{l+m+mi}{1}\PY{p}{]}\PY{o}{\PYZhy{}} \PY{n}{min2}\PY{p}{)}\PY{o}{/}\PY{p}{(}\PY{n}{max2}\PY{o}{\PYZhy{}}\PY{n}{min2}\PY{p}{)}
            \PY{n}{dumbo}\PY{o}{.}\PY{n}{append}\PY{p}{(}\PY{n}{v1}\PY{p}{)}
            \PY{n}{dumbo}\PY{o}{.}\PY{n}{append}\PY{p}{(}\PY{n}{v2}\PY{p}{)}
            \PY{n}{X\PYZus{}init}\PY{o}{.}\PY{n}{append}\PY{p}{(}\PY{n}{dumbo}\PY{p}{)}
        \PY{n}{Xt} \PY{o}{=} \PY{n}{np}\PY{o}{.}\PY{n}{array}\PY{p}{(}\PY{n}{X\PYZus{}init}\PY{p}{[}\PY{n+nb}{len}\PY{p}{(}\PY{n}{X\PYZus{}init}\PY{p}{)}\PY{o}{\PYZhy{}}\PY{l+m+mi}{11}\PY{p}{:}\PY{p}{]}\PY{p}{)}
        \PY{n}{yt} \PY{o}{=} \PY{n}{np}\PY{o}{.}\PY{n}{array}\PY{p}{(}\PY{n}{y\PYZus{}init}\PY{p}{[}\PY{n+nb}{len}\PY{p}{(}\PY{n}{X\PYZus{}init}\PY{p}{)}\PY{o}{\PYZhy{}}\PY{l+m+mi}{11}\PY{p}{:}\PY{p}{]}\PY{p}{)}
        \PY{n}{X} \PY{o}{=} \PY{n}{np}\PY{o}{.}\PY{n}{array}\PY{p}{(}\PY{n}{X\PYZus{}init}\PY{p}{[}\PY{l+m+mi}{0}\PY{p}{:}\PY{n+nb}{len}\PY{p}{(}\PY{n}{X\PYZus{}init}\PY{p}{)}\PY{o}{\PYZhy{}}\PY{l+m+mi}{12}\PY{p}{]}\PY{p}{)}
        \PY{n}{y} \PY{o}{=} \PY{n}{np}\PY{o}{.}\PY{n}{array}\PY{p}{(}\PY{n}{y\PYZus{}init}\PY{p}{[}\PY{l+m+mi}{0}\PY{p}{:}\PY{n+nb}{len}\PY{p}{(}\PY{n}{X\PYZus{}init}\PY{p}{)}\PY{o}{\PYZhy{}}\PY{l+m+mi}{12}\PY{p}{]}\PY{p}{)}
        \PY{n+nb}{print} \PY{p}{(}\PY{n}{yt}\PY{p}{)}
        \PY{n}{plt}\PY{o}{.}\PY{n}{scatter}\PY{p}{(}\PY{n}{X}\PY{p}{[}\PY{p}{:}\PY{p}{,}\PY{l+m+mi}{0}\PY{p}{]}\PY{p}{,} \PY{n}{X}\PY{p}{[}\PY{p}{:}\PY{p}{,}\PY{l+m+mi}{1}\PY{p}{]}\PY{p}{,} \PY{n}{s}\PY{o}{=}\PY{l+m+mi}{40}\PY{p}{,} \PY{n}{c}\PY{o}{=}\PY{n}{y}\PY{p}{,} \PY{n}{cmap}\PY{o}{=}\PY{n}{plt}\PY{o}{.}\PY{n}{cm}\PY{o}{.}\PY{n}{Spectral}\PY{p}{)}
\end{Verbatim}


    \begin{Verbatim}[commandchars=\\\{\}]
[0 0 1 1 0 0 0 1 1 0 0]

    \end{Verbatim}

\begin{Verbatim}[commandchars=\\\{\}]
{\color{outcolor}Out[{\color{outcolor}9}]:} <matplotlib.collections.PathCollection at 0x1e2ed38e518>
\end{Verbatim}
            
    \begin{center}
    \adjustimage{max size={0.9\linewidth}{0.9\paperheight}}{output_3_2.png}
    \end{center}
    { \hspace*{\fill} \\}
    
    There are two labels (healthy sinus and arrythmia) denoted as red and
blue points in the above graph

Our goal is to train an ML classifier that predicts if the patient is
healthy or not.

    \begin{Verbatim}[commandchars=\\\{\}]
{\color{incolor}In [{\color{incolor}10}]:} \PY{c+c1}{\PYZsh{} Helper function to plot a decision boundary.}
         \PY{c+c1}{\PYZsh{} If you don\PYZsq{}t fully understand this function don\PYZsq{}t worry, it just generates the contour plot below.}
         \PY{k}{def} \PY{n+nf}{plot\PYZus{}decision\PYZus{}boundary}\PY{p}{(}\PY{n}{pred\PYZus{}func}\PY{p}{)}\PY{p}{:}
             \PY{c+c1}{\PYZsh{} Set min and max values and give it some padding}
             \PY{n}{x\PYZus{}min}\PY{p}{,} \PY{n}{x\PYZus{}max} \PY{o}{=} \PY{o}{\PYZhy{}}\PY{l+m+mi}{2}\PY{p}{,} \PY{l+m+mi}{4}
             \PY{n}{y\PYZus{}min}\PY{p}{,} \PY{n}{y\PYZus{}max} \PY{o}{=} \PY{l+m+mi}{0}\PY{p}{,} \PY{l+m+mi}{50}
             \PY{n}{h} \PY{o}{=} \PY{l+m+mf}{0.1}
             \PY{c+c1}{\PYZsh{} Generate a grid of points with distance h between them}
             \PY{n}{xx}\PY{p}{,} \PY{n}{yy} \PY{o}{=} \PY{n}{np}\PY{o}{.}\PY{n}{meshgrid}\PY{p}{(}\PY{n}{np}\PY{o}{.}\PY{n}{arange}\PY{p}{(}\PY{n}{x\PYZus{}min}\PY{p}{,} \PY{n}{x\PYZus{}max}\PY{p}{,} \PY{n}{h}\PY{p}{)}\PY{p}{,} \PY{n}{np}\PY{o}{.}\PY{n}{arange}\PY{p}{(}\PY{n}{y\PYZus{}min}\PY{p}{,} \PY{n}{y\PYZus{}max}\PY{p}{,} \PY{n}{h}\PY{p}{)}\PY{p}{)}
             \PY{c+c1}{\PYZsh{} Predict the function value for the whole gid}
             \PY{n}{Z} \PY{o}{=} \PY{n}{pred\PYZus{}func}\PY{p}{(}\PY{n}{np}\PY{o}{.}\PY{n}{c\PYZus{}}\PY{p}{[}\PY{n}{xx}\PY{o}{.}\PY{n}{ravel}\PY{p}{(}\PY{p}{)}\PY{p}{,} \PY{n}{yy}\PY{o}{.}\PY{n}{ravel}\PY{p}{(}\PY{p}{)}\PY{p}{]}\PY{p}{)}
             \PY{n}{Z} \PY{o}{=} \PY{n}{Z}\PY{o}{.}\PY{n}{reshape}\PY{p}{(}\PY{n}{xx}\PY{o}{.}\PY{n}{shape}\PY{p}{)}
             \PY{c+c1}{\PYZsh{} Plot the contour and training examples}
             \PY{n}{figure}\PY{p}{(}\PY{n}{num}\PY{o}{=}\PY{k+kc}{None}\PY{p}{,} \PY{n}{figsize}\PY{o}{=}\PY{p}{(}\PY{l+m+mi}{7}\PY{p}{,} \PY{l+m+mi}{5}\PY{p}{)}\PY{p}{,} \PY{n}{dpi}\PY{o}{=}\PY{l+m+mi}{80}\PY{p}{,} \PY{n}{facecolor}\PY{o}{=}\PY{l+s+s1}{\PYZsq{}}\PY{l+s+s1}{w}\PY{l+s+s1}{\PYZsq{}}\PY{p}{,} \PY{n}{edgecolor}\PY{o}{=}\PY{l+s+s1}{\PYZsq{}}\PY{l+s+s1}{k}\PY{l+s+s1}{\PYZsq{}}\PY{p}{)}
             \PY{n}{plt}\PY{o}{.}\PY{n}{contourf}\PY{p}{(}\PY{n}{xx}\PY{p}{,} \PY{n}{yy}\PY{p}{,} \PY{n}{Z}\PY{p}{,} \PY{n}{cmap}\PY{o}{=}\PY{n}{plt}\PY{o}{.}\PY{n}{cm}\PY{o}{.}\PY{n}{Spectral}\PY{p}{)}
             \PY{n}{plt}\PY{o}{.}\PY{n}{scatter}\PY{p}{(}\PY{n}{X}\PY{p}{[}\PY{p}{:}\PY{p}{,} \PY{l+m+mi}{0}\PY{p}{]}\PY{p}{,} \PY{n}{X}\PY{p}{[}\PY{p}{:}\PY{p}{,} \PY{l+m+mi}{1}\PY{p}{]}\PY{p}{,} \PY{n}{c}\PY{o}{=}\PY{n}{y}\PY{p}{,} \PY{n}{cmap}\PY{o}{=}\PY{n}{plt}\PY{o}{.}\PY{n}{cm}\PY{o}{.}\PY{n}{Spectral}\PY{p}{)}
             \PY{n}{plt}\PY{o}{.}\PY{n}{ylim}\PY{p}{(}\PY{l+m+mi}{0}\PY{p}{,}\PY{l+m+mi}{2}\PY{p}{)}
             \PY{n}{plt}\PY{o}{.}\PY{n}{xlim}\PY{p}{(}\PY{o}{\PYZhy{}}\PY{l+m+mf}{0.1}\PY{p}{,}\PY{l+m+mi}{1}\PY{p}{)}
\end{Verbatim}


    \hypertarget{network-architecture-and-training}{%
\subsection{Network Architecture and
Training}\label{network-architecture-and-training}}

    Firstly, we will build our 3 layer network, with one input layer
(dimensionality = 2 ; PSD \& Energy), hidden layer (dimensionality = 16
; which corrosponds to the perceptrons) and the output layer which takes
from the hidden layer the inputs and maps them to our classes(2: Normal
Wave or not). Furthermore, if we need to expand our network, our
architecture is modular and can be scaled to occupy more classes and
feautures.

    The choice of the number of perceptrons in the hidden layer reflects the
complexity and the ability to comprehend more feautures. Moreover, there
is a tradeoff for chosing high dimensionality in the hidden layer, which
is the computional cost. In spiderman, there was a saying ``With great
power comes great responsibility''; and that exactly what happens with a
neural network with more power(perceptrons), comes greater
responsibility(computional cost).

    Secondly, in order to transfer the output of the layers to closer
values, we need activation functions and specially non-linear
activiation functions. Some of the most important and used activation
functions are the tanh,RELU, sigmoid and softmax.

    \hypertarget{anns-prediction}{%
\subsubsection{ANN's Prediction}\label{anns-prediction}}

Our network makes predictions using \emph{forward propagation}, which is
basically matrix multiplications and the application of the activation
functions we mentioned earlier. If \(x\) is the 2-dimensional input(PSD
\& Energy) to our network, then we compute our prediction \(\hat{y}\)
(also two-dimensional) as follows:

    \[
\begin{aligned}
z_1 & = xW_1 + b_1 \\
a_1 & = \tanh(z_1) \\
z_2 & = a_1W_2 + b_2 \\
a_2 & = \hat{y} = \mathrm{softmax}(z_2)
\end{aligned}
\]

    \(z_i\) is the weighted sum of inputs of layer \(i\) (bias included) and
\(a_i\) is the output of layer \(i\) after applying the activation
function. \(W_1, b_1, W_2, b_2\) are parameters of our network, which we
need to learn from our training data. You can think of them as matrices
transforming data between layers of the network.

    \hypertarget{learning-the-parameters}{%
\subsubsection{Learning the Parameters}\label{learning-the-parameters}}

Learning the parameters for our network means finding parameters
(\(W_1, b_1, W_2, b_2\)) that minimize the error on our training data.
But how do we define the error? We call the function that measures our
error the \emph{loss function}. A common choice with the softmax output
is the
\href{https://en.wikipedia.org/wiki/Cross_entropy\#Cross-entropy_error_function_and_logistic_regression}{cross-entropy
loss}. If we have \(N\) training examples and \(C\) classes then the
loss for our prediction \(\hat{y}\) with respect to the true labels
\(y\) is given by:

\[
\begin{aligned}
L(y,\hat{y}) = - \frac{1}{N} \sum_{n \in N} \sum_{i \in C} y_{n,i} \log\hat{y}_{n,i}
\end{aligned}
\]

    The equation basically calculates the error by summing the values of the
test data and approximates the error, by checking the wrongly classified
points.

    Our objective is to minimize the loss function mentioned earlier, thus
we will use a simple gradient descent algorithm. Moreover, we will need
to define the below equation to be used.

    \[
\begin{aligned}
& \delta_3 = \hat{y} - y \\
& \delta_2 = (1 - \tanh^2z_1) \circ \delta_3W_2^T \\
& \frac{\partial{L}}{\partial{W_2}} = a_1^T \delta_3  \\
& \frac{\partial{L}}{\partial{b_2}} = \delta_3\\
& \frac{\partial{L}}{\partial{W_1}} = x^T \delta_2\\
& \frac{\partial{L}}{\partial{b_1}} = \delta_2 \\
\end{aligned}
\]

    \hypertarget{implementation}{%
\subsubsection{Implementation}\label{implementation}}

Now we are ready for our implementation. We start by defining some
useful variables and parameters for gradient descent:

    \begin{Verbatim}[commandchars=\\\{\}]
{\color{incolor}In [{\color{incolor}11}]:} \PY{n}{num\PYZus{}examples} \PY{o}{=} \PY{n+nb}{len}\PY{p}{(}\PY{n}{X}\PY{p}{)} \PY{c+c1}{\PYZsh{} training set size}
         \PY{n}{nn\PYZus{}input\PYZus{}dim} \PY{o}{=} \PY{l+m+mi}{2} \PY{c+c1}{\PYZsh{} input layer dimensionality}
         \PY{n}{nn\PYZus{}output\PYZus{}dim} \PY{o}{=} \PY{l+m+mi}{2} \PY{c+c1}{\PYZsh{} output layer dimensionality}
         
         \PY{c+c1}{\PYZsh{} Gradient descent parameters (I picked these by hand)}
         \PY{n}{epsilon} \PY{o}{=} \PY{l+m+mf}{0.01} \PY{c+c1}{\PYZsh{} learning rate for gradient descent}
         \PY{n}{reg\PYZus{}lambda} \PY{o}{=} \PY{l+m+mf}{0.01} \PY{c+c1}{\PYZsh{} regularization strength}
\end{Verbatim}


    First let's implement the loss function we defined above. We use this to
evaluate how well our model is doing:

    \begin{Verbatim}[commandchars=\\\{\}]
{\color{incolor}In [{\color{incolor}12}]:} \PY{c+c1}{\PYZsh{} Helper function to evaluate the total loss on the dataset}
         \PY{k}{def} \PY{n+nf}{calculate\PYZus{}loss}\PY{p}{(}\PY{n}{model}\PY{p}{)}\PY{p}{:}
             \PY{n}{W1}\PY{p}{,} \PY{n}{b1}\PY{p}{,} \PY{n}{W2}\PY{p}{,} \PY{n}{b2} \PY{o}{=} \PY{n}{model}\PY{p}{[}\PY{l+s+s1}{\PYZsq{}}\PY{l+s+s1}{W1}\PY{l+s+s1}{\PYZsq{}}\PY{p}{]}\PY{p}{,} \PY{n}{model}\PY{p}{[}\PY{l+s+s1}{\PYZsq{}}\PY{l+s+s1}{b1}\PY{l+s+s1}{\PYZsq{}}\PY{p}{]}\PY{p}{,} \PY{n}{model}\PY{p}{[}\PY{l+s+s1}{\PYZsq{}}\PY{l+s+s1}{W2}\PY{l+s+s1}{\PYZsq{}}\PY{p}{]}\PY{p}{,} \PY{n}{model}\PY{p}{[}\PY{l+s+s1}{\PYZsq{}}\PY{l+s+s1}{b2}\PY{l+s+s1}{\PYZsq{}}\PY{p}{]}
             \PY{c+c1}{\PYZsh{} Forward propagation to calculate our predictions}
             \PY{n}{z1} \PY{o}{=} \PY{n}{X}\PY{o}{.}\PY{n}{dot}\PY{p}{(}\PY{n}{W1}\PY{p}{)} \PY{o}{+} \PY{n}{b1}
             \PY{n}{a1} \PY{o}{=} \PY{n}{np}\PY{o}{.}\PY{n}{tanh}\PY{p}{(}\PY{n}{z1}\PY{p}{)}
             \PY{n}{z2} \PY{o}{=} \PY{n}{a1}\PY{o}{.}\PY{n}{dot}\PY{p}{(}\PY{n}{W2}\PY{p}{)} \PY{o}{+} \PY{n}{b2}
             \PY{n}{exp\PYZus{}scores} \PY{o}{=} \PY{n}{np}\PY{o}{.}\PY{n}{exp}\PY{p}{(}\PY{n}{z2}\PY{p}{)}
             \PY{n}{probs} \PY{o}{=} \PY{n}{exp\PYZus{}scores} \PY{o}{/} \PY{n}{np}\PY{o}{.}\PY{n}{sum}\PY{p}{(}\PY{n}{exp\PYZus{}scores}\PY{p}{,} \PY{n}{axis}\PY{o}{=}\PY{l+m+mi}{1}\PY{p}{,} \PY{n}{keepdims}\PY{o}{=}\PY{k+kc}{True}\PY{p}{)}
             \PY{c+c1}{\PYZsh{} Calculating the loss}
             \PY{n}{corect\PYZus{}logprobs} \PY{o}{=} \PY{o}{\PYZhy{}}\PY{n}{np}\PY{o}{.}\PY{n}{log}\PY{p}{(}\PY{n}{probs}\PY{p}{[}\PY{n+nb}{range}\PY{p}{(}\PY{n}{num\PYZus{}examples}\PY{p}{)}\PY{p}{,} \PY{n}{y}\PY{p}{]}\PY{p}{)}
             \PY{n}{data\PYZus{}loss} \PY{o}{=} \PY{n}{np}\PY{o}{.}\PY{n}{sum}\PY{p}{(}\PY{n}{corect\PYZus{}logprobs}\PY{p}{)}
             \PY{c+c1}{\PYZsh{} Add regulatization term to loss (optional)}
             \PY{n}{data\PYZus{}loss} \PY{o}{+}\PY{o}{=} \PY{n}{reg\PYZus{}lambda}\PY{o}{/}\PY{l+m+mi}{2} \PY{o}{*} \PY{p}{(}\PY{n}{np}\PY{o}{.}\PY{n}{sum}\PY{p}{(}\PY{n}{np}\PY{o}{.}\PY{n}{square}\PY{p}{(}\PY{n}{W1}\PY{p}{)}\PY{p}{)} \PY{o}{+} \PY{n}{np}\PY{o}{.}\PY{n}{sum}\PY{p}{(}\PY{n}{np}\PY{o}{.}\PY{n}{square}\PY{p}{(}\PY{n}{W2}\PY{p}{)}\PY{p}{)}\PY{p}{)}
             \PY{k}{return} \PY{l+m+mf}{1.}\PY{o}{/}\PY{n}{num\PYZus{}examples} \PY{o}{*} \PY{n}{data\PYZus{}loss}
\end{Verbatim}


    We also implement a helper function to calculate the output of the
network. It does forward propagation as defined above and returns the
class with the highest probability.

    \begin{Verbatim}[commandchars=\\\{\}]
{\color{incolor}In [{\color{incolor}13}]:} \PY{c+c1}{\PYZsh{} Helper function to predict an output (0 or 1)}
         \PY{k}{def} \PY{n+nf}{predict}\PY{p}{(}\PY{n}{model}\PY{p}{,} \PY{n}{x}\PY{p}{)}\PY{p}{:}
             \PY{n}{W1}\PY{p}{,} \PY{n}{b1}\PY{p}{,} \PY{n}{W2}\PY{p}{,} \PY{n}{b2} \PY{o}{=} \PY{n}{model}\PY{p}{[}\PY{l+s+s1}{\PYZsq{}}\PY{l+s+s1}{W1}\PY{l+s+s1}{\PYZsq{}}\PY{p}{]}\PY{p}{,} \PY{n}{model}\PY{p}{[}\PY{l+s+s1}{\PYZsq{}}\PY{l+s+s1}{b1}\PY{l+s+s1}{\PYZsq{}}\PY{p}{]}\PY{p}{,} \PY{n}{model}\PY{p}{[}\PY{l+s+s1}{\PYZsq{}}\PY{l+s+s1}{W2}\PY{l+s+s1}{\PYZsq{}}\PY{p}{]}\PY{p}{,} \PY{n}{model}\PY{p}{[}\PY{l+s+s1}{\PYZsq{}}\PY{l+s+s1}{b2}\PY{l+s+s1}{\PYZsq{}}\PY{p}{]}
             \PY{c+c1}{\PYZsh{} Forward propagation}
             \PY{n}{z1} \PY{o}{=} \PY{n}{x}\PY{o}{.}\PY{n}{dot}\PY{p}{(}\PY{n}{W1}\PY{p}{)} \PY{o}{+} \PY{n}{b1}
             \PY{n}{a1} \PY{o}{=} \PY{n}{np}\PY{o}{.}\PY{n}{tanh}\PY{p}{(}\PY{n}{z1}\PY{p}{)}
             \PY{n}{z2} \PY{o}{=} \PY{n}{a1}\PY{o}{.}\PY{n}{dot}\PY{p}{(}\PY{n}{W2}\PY{p}{)} \PY{o}{+} \PY{n}{b2}
             \PY{n}{exp\PYZus{}scores} \PY{o}{=} \PY{n}{np}\PY{o}{.}\PY{n}{exp}\PY{p}{(}\PY{n}{z2}\PY{p}{)}
             \PY{n}{probs} \PY{o}{=} \PY{n}{exp\PYZus{}scores} \PY{o}{/} \PY{n}{np}\PY{o}{.}\PY{n}{sum}\PY{p}{(}\PY{n}{exp\PYZus{}scores}\PY{p}{,} \PY{n}{axis}\PY{o}{=}\PY{l+m+mi}{1}\PY{p}{,} \PY{n}{keepdims}\PY{o}{=}\PY{k+kc}{True}\PY{p}{)}
             \PY{c+c1}{\PYZsh{}print(\PYZdq{}Probs: \PYZbs{}n\PYZdq{},probs)}
             \PY{k}{return} \PY{n}{np}\PY{o}{.}\PY{n}{argmax}\PY{p}{(}\PY{n}{probs}\PY{p}{,} \PY{n}{axis}\PY{o}{=}\PY{l+m+mi}{1}\PY{p}{)}
\end{Verbatim}


    \begin{Verbatim}[commandchars=\\\{\}]
{\color{incolor}In [{\color{incolor}58}]:} \PY{c+c1}{\PYZsh{} This function learns parameters for the neural network and returns the model.}
         \PY{c+c1}{\PYZsh{} \PYZhy{} nn\PYZus{}hdim: Number of nodes in the hidden layer}
         \PY{c+c1}{\PYZsh{} \PYZhy{} num\PYZus{}passes: Number of passes through the training data for gradient descent}
         \PY{c+c1}{\PYZsh{} \PYZhy{} print\PYZus{}loss: If True, print the loss every 20000 iterations}
         \PY{k}{def} \PY{n+nf}{build\PYZus{}model}\PY{p}{(}\PY{n}{nn\PYZus{}hdim}\PY{p}{,} \PY{n}{num\PYZus{}passes}\PY{o}{=}\PY{l+m+mi}{300}\PY{p}{,} \PY{n}{print\PYZus{}loss}\PY{o}{=}\PY{k+kc}{False}\PY{p}{)}\PY{p}{:}
             
             \PY{c+c1}{\PYZsh{} Initialize the parameters to random values. We need to learn these.}
             \PY{n}{np}\PY{o}{.}\PY{n}{random}\PY{o}{.}\PY{n}{seed}\PY{p}{(}\PY{l+m+mi}{1}\PY{p}{)}
             \PY{n}{W1} \PY{o}{=} \PY{n}{np}\PY{o}{.}\PY{n}{random}\PY{o}{.}\PY{n}{randn}\PY{p}{(}\PY{n}{nn\PYZus{}input\PYZus{}dim}\PY{p}{,} \PY{n}{nn\PYZus{}hdim}\PY{p}{)} \PY{o}{/} \PY{n}{np}\PY{o}{.}\PY{n}{sqrt}\PY{p}{(}\PY{n}{nn\PYZus{}input\PYZus{}dim}\PY{p}{)}
             \PY{n}{b1} \PY{o}{=} \PY{n}{np}\PY{o}{.}\PY{n}{zeros}\PY{p}{(}\PY{p}{(}\PY{l+m+mi}{1}\PY{p}{,} \PY{n}{nn\PYZus{}hdim}\PY{p}{)}\PY{p}{)}
             \PY{n}{W2} \PY{o}{=} \PY{n}{np}\PY{o}{.}\PY{n}{random}\PY{o}{.}\PY{n}{randn}\PY{p}{(}\PY{n}{nn\PYZus{}hdim}\PY{p}{,} \PY{n}{nn\PYZus{}output\PYZus{}dim}\PY{p}{)} \PY{o}{/} \PY{n}{np}\PY{o}{.}\PY{n}{sqrt}\PY{p}{(}\PY{n}{nn\PYZus{}hdim}\PY{p}{)}
             \PY{n}{b2} \PY{o}{=} \PY{n}{np}\PY{o}{.}\PY{n}{zeros}\PY{p}{(}\PY{p}{(}\PY{l+m+mi}{1}\PY{p}{,} \PY{n}{nn\PYZus{}output\PYZus{}dim}\PY{p}{)}\PY{p}{)}
         
             \PY{c+c1}{\PYZsh{} This is what we return at the end}
             \PY{n}{model} \PY{o}{=} \PY{p}{\PYZob{}}\PY{p}{\PYZcb{}}
             
             \PY{c+c1}{\PYZsh{} Gradient descent. For each batch...}
             \PY{k}{for} \PY{n}{i} \PY{o+ow}{in} \PY{n+nb}{range}\PY{p}{(}\PY{l+m+mi}{0}\PY{p}{,} \PY{n}{num\PYZus{}passes}\PY{p}{)}\PY{p}{:}
         
                 \PY{c+c1}{\PYZsh{} Forward propagation}
                 \PY{n}{z1} \PY{o}{=} \PY{n}{X}\PY{o}{.}\PY{n}{dot}\PY{p}{(}\PY{n}{W1}\PY{p}{)} \PY{o}{+} \PY{n}{b1}
                 \PY{n}{a1} \PY{o}{=} \PY{n}{np}\PY{o}{.}\PY{n}{tanh}\PY{p}{(}\PY{n}{z1}\PY{p}{)}
                 \PY{n}{z2} \PY{o}{=} \PY{n}{a1}\PY{o}{.}\PY{n}{dot}\PY{p}{(}\PY{n}{W2}\PY{p}{)} \PY{o}{+} \PY{n}{b2}
                 \PY{n}{exp\PYZus{}scores} \PY{o}{=} \PY{n}{np}\PY{o}{.}\PY{n}{exp}\PY{p}{(}\PY{n}{z2}\PY{p}{)}
                 \PY{n}{probs} \PY{o}{=} \PY{n}{exp\PYZus{}scores} \PY{o}{/} \PY{n}{np}\PY{o}{.}\PY{n}{sum}\PY{p}{(}\PY{n}{exp\PYZus{}scores}\PY{p}{,} \PY{n}{axis}\PY{o}{=}\PY{l+m+mi}{1}\PY{p}{,} \PY{n}{keepdims}\PY{o}{=}\PY{k+kc}{True}\PY{p}{)}
         
                 \PY{c+c1}{\PYZsh{} Backpropagation}
                 \PY{n}{delta3} \PY{o}{=} \PY{n}{probs}
                 \PY{n}{delta3}\PY{p}{[}\PY{n+nb}{range}\PY{p}{(}\PY{n}{num\PYZus{}examples}\PY{p}{)}\PY{p}{,} \PY{n}{y}\PY{p}{]} \PY{o}{\PYZhy{}}\PY{o}{=} \PY{l+m+mi}{1}
                 \PY{n}{dW2} \PY{o}{=} \PY{p}{(}\PY{n}{a1}\PY{o}{.}\PY{n}{T}\PY{p}{)}\PY{o}{.}\PY{n}{dot}\PY{p}{(}\PY{n}{delta3}\PY{p}{)}
                 \PY{n}{db2} \PY{o}{=} \PY{n}{np}\PY{o}{.}\PY{n}{sum}\PY{p}{(}\PY{n}{delta3}\PY{p}{,} \PY{n}{axis}\PY{o}{=}\PY{l+m+mi}{0}\PY{p}{,} \PY{n}{keepdims}\PY{o}{=}\PY{k+kc}{True}\PY{p}{)}
                 \PY{n}{delta2} \PY{o}{=} \PY{n}{delta3}\PY{o}{.}\PY{n}{dot}\PY{p}{(}\PY{n}{W2}\PY{o}{.}\PY{n}{T}\PY{p}{)} \PY{o}{*} \PY{p}{(}\PY{l+m+mi}{1} \PY{o}{\PYZhy{}} \PY{n}{np}\PY{o}{.}\PY{n}{power}\PY{p}{(}\PY{n}{a1}\PY{p}{,} \PY{l+m+mi}{2}\PY{p}{)}\PY{p}{)}
                 \PY{n}{dW1} \PY{o}{=} \PY{n}{np}\PY{o}{.}\PY{n}{dot}\PY{p}{(}\PY{n}{X}\PY{o}{.}\PY{n}{T}\PY{p}{,} \PY{n}{delta2}\PY{p}{)}
                 \PY{n}{db1} \PY{o}{=} \PY{n}{np}\PY{o}{.}\PY{n}{sum}\PY{p}{(}\PY{n}{delta2}\PY{p}{,} \PY{n}{axis}\PY{o}{=}\PY{l+m+mi}{0}\PY{p}{)}
         
                 \PY{c+c1}{\PYZsh{} Add regularization terms (b1 and b2 don\PYZsq{}t have regularization terms)}
                 \PY{n}{dW2} \PY{o}{+}\PY{o}{=} \PY{n}{reg\PYZus{}lambda} \PY{o}{*} \PY{n}{W2}
                 \PY{n}{dW1} \PY{o}{+}\PY{o}{=} \PY{n}{reg\PYZus{}lambda} \PY{o}{*} \PY{n}{W1}
         
                 \PY{c+c1}{\PYZsh{} Gradient descent parameter update}
                 \PY{n}{W1} \PY{o}{+}\PY{o}{=} \PY{o}{\PYZhy{}}\PY{n}{epsilon} \PY{o}{*} \PY{n}{dW1}
                 \PY{n}{b1} \PY{o}{+}\PY{o}{=} \PY{o}{\PYZhy{}}\PY{n}{epsilon} \PY{o}{*} \PY{n}{db1}
                 \PY{n}{W2} \PY{o}{+}\PY{o}{=} \PY{o}{\PYZhy{}}\PY{n}{epsilon} \PY{o}{*} \PY{n}{dW2}
                 \PY{n}{b2} \PY{o}{+}\PY{o}{=} \PY{o}{\PYZhy{}}\PY{n}{epsilon} \PY{o}{*} \PY{n}{db2}
                 
                 \PY{c+c1}{\PYZsh{} Assign new parameters to the model}
                 \PY{n}{model} \PY{o}{=} \PY{p}{\PYZob{}} \PY{l+s+s1}{\PYZsq{}}\PY{l+s+s1}{W1}\PY{l+s+s1}{\PYZsq{}}\PY{p}{:} \PY{n}{W1}\PY{p}{,} \PY{l+s+s1}{\PYZsq{}}\PY{l+s+s1}{b1}\PY{l+s+s1}{\PYZsq{}}\PY{p}{:} \PY{n}{b1}\PY{p}{,} \PY{l+s+s1}{\PYZsq{}}\PY{l+s+s1}{W2}\PY{l+s+s1}{\PYZsq{}}\PY{p}{:} \PY{n}{W2}\PY{p}{,} \PY{l+s+s1}{\PYZsq{}}\PY{l+s+s1}{b2}\PY{l+s+s1}{\PYZsq{}}\PY{p}{:} \PY{n}{b2}\PY{p}{\PYZcb{}}
                 
                 \PY{c+c1}{\PYZsh{} Optionally print the loss.}
                 \PY{n}{best} \PY{o}{=} \PY{l+m+mi}{1000}
                 \PY{n}{losses}\PY{o}{=}\PY{n}{calculate\PYZus{}loss}\PY{p}{(}\PY{n}{model}\PY{p}{)}
                 \PY{k}{if} \PY{n}{best}\PY{o}{\PYZgt{}}\PY{n}{losses}\PY{p}{:}
                     \PY{n}{best} \PY{o}{=} \PY{n}{losses}
                     \PY{n}{BW1} \PY{o}{=} \PY{n}{W1}
                     \PY{n}{Bb1} \PY{o}{=} \PY{n}{b1}
                     \PY{n}{BW2} \PY{o}{=} \PY{n}{W2}
                     \PY{n}{Bb2} \PY{o}{=} \PY{n}{b2}
                 \PY{n}{W1A} \PY{o}{=} \PY{p}{[}\PY{p}{]}
                 \PY{n}{W2A} \PY{o}{=} \PY{p}{[}\PY{p}{]}
                 \PY{n}{b1A} \PY{o}{=} \PY{p}{[}\PY{p}{]}
                 \PY{n}{b2A} \PY{o}{=} \PY{p}{[}\PY{p}{]}
                 \PY{k}{for} \PY{n}{i} \PY{o+ow}{in} \PY{n}{Bb1}\PY{p}{:}
                     \PY{k}{for} \PY{n}{j} \PY{o+ow}{in} \PY{n}{i}\PY{p}{:}
                         \PY{n}{b1A}\PY{o}{.}\PY{n}{append}\PY{p}{(}\PY{n}{j}\PY{p}{)}
                 \PY{k}{for} \PY{n}{i} \PY{o+ow}{in} \PY{n}{Bb2}\PY{p}{:}
                     \PY{k}{for} \PY{n}{j} \PY{o+ow}{in} \PY{n}{i}\PY{p}{:}
                         \PY{n}{b2A}\PY{o}{.}\PY{n}{append}\PY{p}{(}\PY{n}{j}\PY{p}{)}
                 \PY{k}{for} \PY{n}{j} \PY{o+ow}{in} \PY{n}{BW1}\PY{p}{:}
                     \PY{k}{for} \PY{n}{i} \PY{o+ow}{in} \PY{n}{j}\PY{p}{:}
                         \PY{n}{W1A}\PY{o}{.}\PY{n}{append}\PY{p}{(}\PY{n}{i}\PY{p}{)}
                 \PY{k}{for} \PY{n}{j} \PY{o+ow}{in} \PY{n}{BW2}\PY{p}{:}
                     \PY{k}{for} \PY{n}{i} \PY{o+ow}{in} \PY{n}{j}\PY{p}{:}
                         \PY{n}{W2A}\PY{o}{.}\PY{n}{append}\PY{p}{(}\PY{n}{i}\PY{p}{)}
                 \PY{n}{b1A} \PY{o}{=} \PY{n}{np}\PY{o}{.}\PY{n}{array}\PY{p}{(}\PY{n}{b1A}\PY{p}{)}
                 \PY{n}{b2A} \PY{o}{=} \PY{n}{np}\PY{o}{.}\PY{n}{array}\PY{p}{(}\PY{n}{b2A}\PY{p}{)}
                 \PY{n}{W2A} \PY{o}{=} \PY{n}{np}\PY{o}{.}\PY{n}{array}\PY{p}{(}\PY{n}{W2A}\PY{p}{)}
                 \PY{n}{W1A} \PY{o}{=} \PY{n}{np}\PY{o}{.}\PY{n}{array}\PY{p}{(}\PY{n}{W1A}\PY{p}{)}
                 \PY{n}{np}\PY{o}{.}\PY{n}{savetxt}\PY{p}{(}\PY{l+s+s2}{\PYZdq{}}\PY{l+s+s2}{bestWeights/BW1.csv}\PY{l+s+s2}{\PYZdq{}}\PY{p}{,} \PY{n}{W1A}\PY{p}{)}
                 \PY{n}{np}\PY{o}{.}\PY{n}{savetxt}\PY{p}{(}\PY{l+s+s2}{\PYZdq{}}\PY{l+s+s2}{bestWeights/b1.csv}\PY{l+s+s2}{\PYZdq{}}\PY{p}{,} \PY{n}{b1A}\PY{p}{)}
                 \PY{n}{np}\PY{o}{.}\PY{n}{savetxt}\PY{p}{(}\PY{l+s+s2}{\PYZdq{}}\PY{l+s+s2}{bestWeights/BW2.csv}\PY{l+s+s2}{\PYZdq{}}\PY{p}{,} \PY{n}{W2A}\PY{p}{)}
                 \PY{n}{np}\PY{o}{.}\PY{n}{savetxt}\PY{p}{(}\PY{l+s+s2}{\PYZdq{}}\PY{l+s+s2}{bestWeights/b2.csv}\PY{l+s+s2}{\PYZdq{}}\PY{p}{,} \PY{n}{b2A}\PY{p}{)}
         \PY{c+c1}{\PYZsh{}     print (\PYZdq{}Weights of edges of Input layer to hidden layer (2x16): \PYZdq{},W1, \PYZdq{}\PYZbs{}n\PYZdq{})}
         \PY{c+c1}{\PYZsh{}     print (\PYZdq{}biases of hidden layer (x16): \PYZdq{},b1, \PYZdq{}\PYZbs{}n\PYZdq{})}
         \PY{c+c1}{\PYZsh{}     print (\PYZdq{}Weights of edges of hidden layer to output layer (16x2): \PYZdq{},W2, \PYZdq{}\PYZbs{}n\PYZdq{})}
         \PY{c+c1}{\PYZsh{}     print (\PYZdq{}biases of output layer (x2): \PYZdq{},b2, \PYZdq{}\PYZbs{}n\PYZdq{})}
             \PY{n+nb}{print}\PY{p}{(}\PY{n}{best}\PY{p}{)}
             \PY{k}{return} \PY{n}{model}
\end{Verbatim}


    \hypertarget{selected-hidden-layer-neurons-size16}{%
\subsubsection{Selected hidden layer neurons
size(16)}\label{selected-hidden-layer-neurons-size16}}

Based on the hidden layer neurons count variant training we chose 16.

    \begin{Verbatim}[commandchars=\\\{\}]
{\color{incolor}In [{\color{incolor}59}]:} \PY{c+c1}{\PYZsh{} Build a model with a 16\PYZhy{}dimensional hidden layer}
         \PY{n}{model} \PY{o}{=} \PY{n}{build\PYZus{}model}\PY{p}{(}\PY{l+m+mi}{16}\PY{p}{,} \PY{n}{print\PYZus{}loss}\PY{o}{=}\PY{k+kc}{True}\PY{p}{)}
\end{Verbatim}


    \begin{Verbatim}[commandchars=\\\{\}]
0.08991451839856202

    \end{Verbatim}

    \begin{Verbatim}[commandchars=\\\{\}]
{\color{incolor}In [{\color{incolor}60}]:} \PY{n}{scores} \PY{o}{=} \PY{l+m+mi}{0}
         \PY{k}{for} \PY{n}{i} \PY{o+ow}{in} \PY{n+nb}{range}\PY{p}{(}\PY{l+m+mi}{0}\PY{p}{,} \PY{l+m+mi}{11}\PY{p}{,} \PY{l+m+mi}{1}\PY{p}{)}\PY{p}{:}
             \PY{k}{if} \PY{n}{predict}\PY{p}{(}\PY{n}{model}\PY{p}{,}\PY{n}{Xt}\PY{p}{[}\PY{n}{i}\PY{p}{]}\PY{p}{)} \PY{o}{==} \PY{n}{yt}\PY{p}{[}\PY{n}{i}\PY{p}{]}\PY{p}{:}
                 \PY{n+nb}{print} \PY{p}{(}\PY{n}{yt}\PY{p}{[}\PY{n}{i}\PY{p}{]}\PY{p}{)}
                 \PY{n}{scores}\PY{o}{+}\PY{o}{=}\PY{l+m+mi}{1}
         \PY{n+nb}{print}\PY{p}{(}\PY{l+s+s2}{\PYZdq{}}\PY{l+s+s2}{the accuracy is : }\PY{l+s+s2}{\PYZdq{}}\PY{p}{,}\PY{n}{scores}\PY{o}{/}\PY{l+m+mi}{10}\PY{p}{)}
         \PY{c+c1}{\PYZsh{} Plot the decision boundary}
         \PY{n}{plot\PYZus{}decision\PYZus{}boundary}\PY{p}{(}\PY{k}{lambda} \PY{n}{x}\PY{p}{:} \PY{n}{predict}\PY{p}{(}\PY{n}{model}\PY{p}{,} \PY{n}{x}\PY{p}{)}\PY{p}{)}
         \PY{n}{plt}\PY{o}{.}\PY{n}{title}\PY{p}{(}\PY{l+s+s2}{\PYZdq{}}\PY{l+s+s2}{Decision Boundary for hidden layer size 16}\PY{l+s+s2}{\PYZdq{}}\PY{p}{)}
\end{Verbatim}


    \begin{Verbatim}[commandchars=\\\{\}]
0
1
1
0
0
0
1
1
the accuracy is :  0.8

    \end{Verbatim}

\begin{Verbatim}[commandchars=\\\{\}]
{\color{outcolor}Out[{\color{outcolor}60}]:} Text(0.5,1,'Decision Boundary for hidden layer size 16')
\end{Verbatim}
            
    \begin{center}
    \adjustimage{max size={0.9\linewidth}{0.9\paperheight}}{output_26_2.png}
    \end{center}
    { \hspace*{\fill} \\}
    
    Varying the hidden layer size

In the example above we picked a hidden layer size of 16. Let's now get
a sense of how varying the hidden layer size affects the result.

    \begin{Verbatim}[commandchars=\\\{\}]
{\color{incolor}In [{\color{incolor}65}]:} \PY{n}{hidden\PYZus{}layer\PYZus{}dimensions} \PY{o}{=} \PY{p}{[}\PY{l+m+mi}{1}\PY{p}{,}\PY{l+m+mi}{16}\PY{p}{,}\PY{l+m+mi}{2000}\PY{p}{]}
         \PY{k}{for} \PY{n}{i}\PY{p}{,} \PY{n}{nn\PYZus{}hdim} \PY{o+ow}{in} \PY{n+nb}{enumerate}\PY{p}{(}\PY{n}{hidden\PYZus{}layer\PYZus{}dimensions}\PY{p}{)}\PY{p}{:}
             \PY{n+nb}{print}\PY{p}{(}\PY{l+s+s2}{\PYZdq{}}\PY{l+s+s2}{Model}\PY{l+s+s2}{\PYZdq{}}\PY{p}{,}\PY{n}{i}\PY{o}{+}\PY{l+m+mi}{1}\PY{p}{,}\PY{l+s+s2}{\PYZdq{}}\PY{l+s+s2}{With }\PY{l+s+si}{\PYZpc{}d}\PY{l+s+s2}{ perceptrons}\PY{l+s+se}{\PYZbs{}n}\PY{l+s+s2}{ Loss = }\PY{l+s+s2}{\PYZdq{}}\PY{o}{\PYZpc{}}\PY{k}{nn\PYZus{}hdim},end=\PYZdq{}\PYZdq{})
             \PY{n}{model} \PY{o}{=} \PY{n}{build\PYZus{}model}\PY{p}{(}\PY{n}{nn\PYZus{}hdim}\PY{p}{,} \PY{n}{print\PYZus{}loss}\PY{o}{=}\PY{k+kc}{True}\PY{p}{)}
             \PY{n+nb}{print}\PY{p}{(}\PY{l+s+s1}{\PYZsq{}}\PY{l+s+se}{\PYZbs{}n}\PY{l+s+s1}{\PYZsq{}}\PY{p}{)}
             \PY{n}{plot\PYZus{}decision\PYZus{}boundary}\PY{p}{(}\PY{k}{lambda} \PY{n}{x}\PY{p}{:} \PY{n}{predict}\PY{p}{(}\PY{n}{model}\PY{p}{,} \PY{n}{x}\PY{p}{)}\PY{p}{)}
             \PY{n}{plt}\PY{o}{.}\PY{n}{show}\PY{p}{(}\PY{p}{)}
\end{Verbatim}


    \begin{Verbatim}[commandchars=\\\{\}]
Model 1 With 1 perceptrons
 Loss = 0.09983589374395352



    \end{Verbatim}

    \begin{center}
    \adjustimage{max size={0.9\linewidth}{0.9\paperheight}}{output_28_1.png}
    \end{center}
    { \hspace*{\fill} \\}
    
    \begin{Verbatim}[commandchars=\\\{\}]

















Model 2 With 16 perceptrons
 Loss = 0.08991451839856202



    \end{Verbatim}

    \begin{center}
    \adjustimage{max size={0.9\linewidth}{0.9\paperheight}}{output_28_3.png}
    \end{center}
    { \hspace*{\fill} \\}
    
    \begin{Verbatim}[commandchars=\\\{\}]

















Model 3 With 2000 perceptrons
 Loss = 0.2782942615519288



    \end{Verbatim}

    \begin{center}
    \adjustimage{max size={0.9\linewidth}{0.9\paperheight}}{output_28_5.png}
    \end{center}
    { \hspace*{\fill} \\}
    

    % Add a bibliography block to the postdoc
    
    
    
    \end{document}
